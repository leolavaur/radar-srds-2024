%%%%%%%%%%%%%%%%%%%%%%%%%%%%%%%%%%%%%%%%%%%%%%%%%
%% Text macros                                 %%
%%%%%%%%%%%%%%%%%%%%%%%%%%%%%%%%%%%%%%%%%%%%%%%%%

% Name of the contribution. Replace here to make 
% it effective all over the document.
\newcommand{\thecontrib}{\texttt{Trust-FIDS}\xspace}
\newcommand{\codeurl}{\url{https://github.com/<REDACTED>}}

%%%%%%%%%%%%%%%%%%%%%%%%%%%%%%%%%%%%%%%%%%%%%%%%%
%% Math macros                                 %%
%%%%%%%%%%%%%%%%%%%%%%%%%%%%%%%%%%%%%%%%%%%%%%%%%

% Notes about xparse
% Syntax: \NewDocumentCommand {name} {arguments} {body}
%   name: the name of the command
%   arguments: the arguments of the command, separated by spaces, eg. { o o m }.
%     In the argument list, o means optional, m means mandatory. Arguments can 
%     be accesesd with #1, #2, etc. Optional arguments can be provided with a
%     default value, eg. { O{default} }.
%       
\usepackage{xparse}

% \mathdef is a macro that will be used to define new macros that will be used in
%   math mode, without impacting the default behavior of the macro in text mode.
%   The macro is defined as a command that takes 3 arguments:
%     1. The name of the macro to be defined
%     2. The arguments of the macro to be defined, following the same syntax as
%        defined in xparse
%     3. The body of the macro to be defined

\DeclareDocumentCommand{\mathdef}{ m O{ } m }{%
  \expandafter\let \csname old\string#1\endcsname=#1
  \expandafter\NewDocumentCommand \csname new\string#1\endcsname { #2 }{#3}
  \DeclareRobustCommand #1 {%
    \ifmmode
      \expandafter\let\expandafter\next\csname new\string#1\endcsname
    \else
      \expandafter\let\expandafter\next\csname old\string#1\endcsname
    \fi
    \next
  }%
}

% ----------------------------
% Fedeted Learning
% ----------------------------

% Participant #1, defaults to i
\mathdef{\p}[ O{i} ]{p_{#1}}
% Number of participants, probably useless.
\mathdef{\n}{n}
% Set of all participants, same
\mathdef{\P}{P}
% Local dataset of participant $p_#1$, defaults to i
\mathdef{\d}[ O{i} ]{d_{#1}}
% Union of all local datasets
\mathdef{\D}{D}

% ----------------------------
% Clustering
% ----------------------------

% Cluster $k$ at round $r$
\mathdef{\c}[ O{k} O{r} ]{C_{#1}^{#2}}
% Number of clusters at round $r$
\mathdef{\m}[ O{r} ]{m^{#1}}
% Center of cluster k at round $r$
\mathdef{\center}[ O{k} O{r} ]{\mu_{#1}^{#2}}
% Set of clusters at round $r$
\mathdef{\C}[ O{r} ]{\mathscr{C}^{#1}}
% Distance between $p_i$ and $p_j$
\mathdef{\pdist}[ O{i} O{j} O{r} ]{\delta_{{#1},{#2}}^{#3}}
% Distance from cluster $k$ and $\ell$ centers at round $r$
\mathdef{\kdist}[ O{k} O{\ell} O{r} ]{\Delta_{{#1},{#2}}^{#3}}
\mathdef{\mdist}[ O{r} ]{\overline{\Delta^{#1}}}

\renewcommand{\Pr}{\mathbb{P}}

% ----------------------------
% Models
% ----------------------------

% Local model of participant $i$ at round $r$
\mathdef{\w}[ O{i} O{r} ]{w_{#1}^{#2}}
% Local model of participant $i$ weight in aggregation at round $r$
\mathdef{\weight}[ O{i} O{r} ]{\rho_{#1}^{#2}}
% All local models from participants at round $r$
\mathdef{\W}[ O{r} ]{W^{#1}}
% Global model for cluster $c_k^r$ at round $r$
\mathdef{\wbar}[ O{k} O{r} ]{\overline{w}_{#1}^{#2}}
% All cluster models at round $r$
\mathdef{\Wbar}[ O{r} ]{\overline{W}^r}

% ----------------------------
% Evaluations
% ----------------------------

% Evaluation of $w_j^r$ using $p_i$ local dataset $d_i$
\mathdef{\e}[ O{i} O{j} O{r} ]{e_{{#1},{#2}}^{{#3}}}
% Matrix of all evaluations at round $r$; of size $n \times n$
\mathdef{\E}[ O{r} ]{E^{#1}}
% $p_i$ evaluation on every participant at round $r$
\mathdef{\issue}[ O{i} O{r} ]{E^{#2}_{[{#1},*]}}
% Participants evaluations on $p_j$ at round $r$
\mathdef{\rece}[ O{j} O{r} ]{E^{#2}_{[*,{#1}]}}
% $p_i$ evaluation on every participant at round $r$
\mathdef{\evals}[ O{i} O{j} O{r} ]{E_{{#1},{#2}}^{{#3}}}

% ----------------------------
% Repuation
% ----------------------------
\mathdef{\Prob}[]{\vec{\mathbb{P}}}
\mathdef{\prob}[ O{q} ]{\mathbb{P}\{{#1}\}}
\mathdef{\cond}[ O{s} ]{\mathbb{P}\{\varepsilon_s|\vec{\gamma^r}\}}

\mathdef{\rep}[ O{i} O{r}]{\psi_{#1}^{#2}}


% TODO retester mathbb avec ACM. 