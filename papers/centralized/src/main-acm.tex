\documentclass[
    format=sigplan,
    screen,
    natbib=false,
]{acmart}
% Set page number with acme template : https://tex.stackexchange.com/questions/358088/add-page-number-in-acm-2017-sigconf-template 
\settopmatter{printfolios=true}

%% NOTE that a single column version is required for 
%% submission and peer review. This can be done by changing
%% the \doucmentclass[...]{acmart} in this template to 
%% \documentclass[manuscript,screen,review]{acmart}
%% 
%% To ensure 100% compatibility, please check the white list of
%% approved LaTeX packages to be used with the Master Article Template at
%% https://www.acm.org/publications/taps/whitelist-of-latex-packages 
%% before creating your document. The white list page provides  
%% information on how to submit additional LaTeX packages for 
%% review and adoption.
%% Fonts used in the template cannot be substituted; margin 
%% adjustments are not allowed.
%%
%% \BibTeX command to typeset BibTeX logo in the docs
\AtBeginDocument{%
  \providecommand\BibTeX{{%
    \normalfont B\kern-0.5em{\scshape i\kern-0.25em b}\kern-0.8em\TeX}}}

%% Rights management information. This information is sent to you
%% when you complete the rights form. These commands have SAMPLE
%% values in them; it is your responsibility as an author to replace
%% the commands and values with those provided to you when you
%% complete the rights form.
%\setcopyright{acmcopyright}
%\copyrightyear{2023}
\acmYear{2023}
%\acmDOI{XXXXXXX.XXXXXXX}

%% These commands are for a PROCEEDINGS abstract or paper.
\acmConference[RAID '23]{26th International Symposium on Research in Attacks, Intrusions, and Defenses}{October 16--18, 2023}{Hong Kong, China}

% \acmPrice{15.00}
% \acmISBN{978-1-4503-XXXX-X/23/06}


%% Submission ID.
%% Use this when submitting an article to a sponsored event. You'll
%% receive a unique submission ID from the organizers
%% of the event, and this ID should be used as the parameter to this command.
%%\acmSubmissionID{123-A56-BU3}

%%
%% For managing citations, it is recommended to use bibliography
%% files in BibTeX format.
%%
%% You can then either use BibTeX with the ACM-Reference-Format style,
%% or BibLaTeX with the acmnumeric or acmauthoryear sytles, that include
%% support for advanced citation of software artefact from the
%% biblatex-software package, also separately available on CTAN.
%%
%% Look at the sample-*-biblatex.tex files for templates showcasing
%% the biblatex styles.
%%

%%%%%%%%%%%%%%%%%%%%%%%%%%%%%%%%%%%%%%%%%%%%%%%%%%%%%%%%%%%%%%%%%%%%%%%%%%%%%%%%
%               PACKAGES AND OPTIONS
%%%%%%%%%%%%%%%%%%%%%%%%%%%%%%%%%%%%%%%%%%%%%%%%%%%%%%%%%%%%%%%%%%%%%%%%%%%%%%%%

% Use \PassOptionsToPackage here to pass options to packages already included in 
% acmart.cls.

% Avoiding imports of commonly used packages
% ------------------------------------------------------------------------------

\usepackage{amsmath,amsfonts}         % for math
\usepackage{float}                    % for figures
\usepackage{hyperref}                 % for hyperlinks
\usepackage{booktabs,makecell}        % for tables
 
% Packages for specific use
% ------------------------------------------------------------------------------

% lists and enumerations
\usepackage[inline,shortlabels]{enumitem}
% theorems definitions
\usepackage{amsthm}
% cross-references
\usepackage{cleveref}
\usepackage{suffix}
% abbreviations
\usepackage{xspace}
% balancing columns
\usepackage{balance}
% Graphs and subgraphs
\usepackage{subcaption,graphicx}



% Other packages and options
% ------------------------------------------------------------------------------

%% Glossary
\usepackage[acronym]{glossaries}
\glsdisablehyper % disable the hyperlinks
\makenoidxglossaries

%% Bibliography
\RequirePackage[
    datamodel=acmdatamodel,
    style=acmnumeric,
    natbib,
    % mincitenames=1,
    % maxcitenames=1,
    uniquelist
]{biblatex}

\addbibresource{bibliography.bib}
\addbibresource{trust-fids.bib}
\addbibresource{bib-pm.bib}
\addbibresource{bib-leo.bib}

% Clear useless fields
\AtEveryBibitem{\clearfield{month}}
\AtEveryBibitem{\clearfield{pages}}
\AtEveryBibitem{\clearfield{editor}}
\AtEveryBibitem{\clearfield{publisher}}
\AtEveryBibitem{\clearfield{address}}
\AtEveryBibitem{\clearfield{series}}
\AtEveryBibitem{\clearfield{isbn}}
\AtEveryBibitem{\clearfield{issn}}
\AtEveryBibitem{\clearfield{note}}
\AtEveryBibitem{\clearfield{volume}}
\AtEveryBibitem{\clearfield{number}}
\AtEveryBibitem{\clearfield{location}}
\AtEveryBibitem{\clearfield{eprinttype}}
\AtEveryBibitem{\clearfield{eprint}}

% link DOI or URL if available
\newbibmacro{string+doi}[1]{%
  \iffieldundef{doi}{%
  	\iffieldundef{url}{#1}{\href{\thefield{url}}{#1}}}{\href{http://dx.doi.org/\thefield{doi}}{#1}}}

% on the title, in color
\DeclareFieldFormat
  [article,inbook,incollection,inproceedings,patent,thesis,unpublished,misc,techreport]
  {title}{\usebibmacro{string+doi}{#1\addperiod}}
\AtEveryBibitem{\clearfield{doi}}

% Algorithms
\usepackage{algorithm}
\usepackage{algpseudocodex}

% Tables
\usepackage{tabularx}
\usepackage{multirow}

%Images / svg
\usepackage{svg}

% plot 
\usepackage{pgfplots}
\pgfplotsset{compat=1.18}

%tikz
\usepackage{tikz}
\usetikzlibrary{calc}

% Properly spaced abbreviations, taken from the CVPR's style
% package (https://stackoverflow.com/a/39363004).
% Adds a period to the end of an abbreviation unless there's one
% already, then \xspace.
\makeatletter
\DeclareRobustCommand\onedot{\futurelet\@let@token\@onedot}
\def\@onedot{\ifx\@let@token.\else.\null\fi\xspace}
\def\eg{\emph{e.g}\onedot} \def\Eg{\emph{E.g}\onedot}
\def\ie{\emph{i.e}\onedot} \def\Ie{\emph{I.e}\onedot}
\def\cf{\emph{c.f}\onedot} \def\Cf{\emph{C.f}\onedot}
\def\etc{\emph{etc}\onedot} \def\vs{\emph{vs}\onedot}
\def\wrt{w.r.t\onedot} \def\dof{d.o.f\onedot}
\def\etal{\emph{et al}\onedot}
\makeatother

% \needref command
\newcommand{\needref}{\textbf{[?]}\xspace}

% Special block definition
% ------------------------------------------------------------------------------
% aglorithmic block definitions (https://github.com/chrmatt/algpseudocodex/issues/3)
\algnewcommand\algorithmicwith{\textbf{with}}%

\makeatletter
\algdef{SE}[WITH]{With}{EndWith}[1]{\algpx@startCodeCommand\algpx@startIndent\algorithmicwith\ #1\ \algorithmicdo}{\algorithmicend\ \algorithmicwith}%
\ifbool{algpx@noEnd}{%
  \algtext*{EndWith}%
  %
  % end indent line after (not before), to get correct y position for multiline text in last command
  \apptocmd{\EndWith}{\algpx@endIndent}{}{}%
}{}%

\pretocmd{\With}{\algpx@endCodeCommand}{}{}

% for end commands that may not be printed, tell endCodeCommand whether we are using noEnd
\ifbool{algpx@noEnd}{%
  \pretocmd{\EndWith}{\algpx@endCodeCommand[1]}{}{}%
}{%
  \pretocmd{\EndWith}{\algpx@endCodeCommand[0]}{}{}%
}%
\makeatother

% Personalized theorems
% ------------------------------------------------------------------------------

%% Hypotheses
\theoremstyle{definition}
\newtheorem{hypothesis}{Hypothesis}
\crefname{hypothesis}{hypothesis}{hypotheses}
\Crefname{hypothesis}{Hypothesis}{Hypotheses}

%% Challenges
\theoremstyle{definition}
\newtheorem{challenge}{Challenge}
\crefname{challenge}{challenge}{challenges}
\Crefname{challenge}{Challenge}{Challenges}

%% Research Questions
\newtheorem{innerRQ}{RQ}
\crefname{innerRQ}{RQ}{RQs}
\Crefname{innerRQ}{RQ}{RQs}

\newenvironment{RQ}[1]
  {\renewcommand\theinnerRQ{#1}\innerRQ}
  {\endinnerRQ}

\def\changemargin#1#2{\list{}{\rightmargin#2\leftmargin#1}\item[]}
\let\endchangemargin=\endlist 

% Tables
% ------------------------------------------------------------------------------

\newcommand{\ccell}[1]{\multicolumn{1}{c}{#1}}

% File inputs
% ------------------------------------------------------------------------------

% Notation macros
%%%%%%%%%%%%%%%%%%%%%%%%%%%%%%%%%%%%%%%%%%%%%%%%%
%% Text macros                                 %%
%%%%%%%%%%%%%%%%%%%%%%%%%%%%%%%%%%%%%%%%%%%%%%%%%

% Name of the contribution. Replace here to make 
% it effective all over the document.
\newcommand{\thecontrib}{\texttt{Trust-FIDS}\xspace}
\newcommand{\codeurl}{\url{https://github.com/<REDACTED>}}

%%%%%%%%%%%%%%%%%%%%%%%%%%%%%%%%%%%%%%%%%%%%%%%%%
%% Math macros                                 %%
%%%%%%%%%%%%%%%%%%%%%%%%%%%%%%%%%%%%%%%%%%%%%%%%%

% Notes about xparse
% Syntax: \NewDocumentCommand {name} {arguments} {body}
%   name: the name of the command
%   arguments: the arguments of the command, separated by spaces, eg. { o o m }.
%     In the argument list, o means optional, m means mandatory. Arguments can 
%     be accesesd with #1, #2, etc. Optional arguments can be provided with a
%     default value, eg. { O{default} }.
%       
\usepackage{xparse}

% \mathdef is a macro that will be used to define new macros that will be used in
%   math mode, without impacting the default behavior of the macro in text mode.
%   The macro is defined as a command that takes 3 arguments:
%     1. The name of the macro to be defined
%     2. The arguments of the macro to be defined, following the same syntax as
%        defined in xparse
%     3. The body of the macro to be defined

\DeclareDocumentCommand{\mathdef}{ m O{ } m }{%
  \expandafter\let \csname old\string#1\endcsname=#1
  \expandafter\NewDocumentCommand \csname new\string#1\endcsname { #2 }{#3}
  \DeclareRobustCommand #1 {%
    \ifmmode
      \expandafter\let\expandafter\next\csname new\string#1\endcsname
    \else
      \expandafter\let\expandafter\next\csname old\string#1\endcsname
    \fi
    \next
  }%
}

% ----------------------------
% Fedeted Learning
% ----------------------------

% Participant #1, defaults to i
\mathdef{\p}[ O{i} ]{p_{#1}}
% Number of participants, probably useless.
\mathdef{\n}{n}
% Set of all participants, same
\mathdef{\P}{P}
% Local dataset of participant $p_#1$, defaults to i
\mathdef{\d}[ O{i} ]{d_{#1}}
% Union of all local datasets
\mathdef{\D}{D}

% ----------------------------
% Clustering
% ----------------------------

% Cluster $k$ at round $r$
\mathdef{\c}[ O{k} O{r} ]{C_{#1}^{#2}}
% Number of clusters at round $r$
\mathdef{\m}[ O{r} ]{m^{#1}}
% Center of cluster k at round $r$
\mathdef{\center}[ O{k} O{r} ]{\mu_{#1}^{#2}}
% Set of clusters at round $r$
\mathdef{\C}[ O{r} ]{\mathscr{C}^{#1}}
% Distance between $p_i$ and $p_j$
\mathdef{\pdist}[ O{i} O{j} O{r} ]{\delta_{{#1},{#2}}^{#3}}
% Distance from cluster $k$ and $\ell$ centers at round $r$
\mathdef{\kdist}[ O{k} O{\ell} O{r} ]{\Delta_{{#1},{#2}}^{#3}}
\mathdef{\mdist}[ O{r} ]{\overline{\Delta^{#1}}}

\renewcommand{\Pr}{\mathbb{P}}

% ----------------------------
% Models
% ----------------------------

% Local model of participant $i$ at round $r$
\mathdef{\w}[ O{i} O{r} ]{w_{#1}^{#2}}
% Local model of participant $i$ weight in aggregation at round $r$
\mathdef{\weight}[ O{i} O{r} ]{\rho_{#1}^{#2}}
% All local models from participants at round $r$
\mathdef{\W}[ O{r} ]{W^{#1}}
% Global model for cluster $c_k^r$ at round $r$
\mathdef{\wbar}[ O{k} O{r} ]{\overline{w}_{#1}^{#2}}
% All cluster models at round $r$
\mathdef{\Wbar}[ O{r} ]{\overline{W}^r}

% ----------------------------
% Evaluations
% ----------------------------

% Evaluation of $w_j^r$ using $p_i$ local dataset $d_i$
\mathdef{\e}[ O{i} O{j} O{r} ]{e_{{#1},{#2}}^{{#3}}}
% Matrix of all evaluations at round $r$; of size $n \times n$
\mathdef{\E}[ O{r} ]{E^{#1}}
% $p_i$ evaluation on every participant at round $r$
\mathdef{\issue}[ O{i} O{r} ]{E^{#2}_{[{#1},*]}}
% Participants evaluations on $p_j$ at round $r$
\mathdef{\rece}[ O{j} O{r} ]{E^{#2}_{[*,{#1}]}}
% $p_i$ evaluation on every participant at round $r$
\mathdef{\evals}[ O{i} O{j} O{r} ]{E_{{#1},{#2}}^{{#3}}}

% ----------------------------
% Repuation
% ----------------------------
\mathdef{\Prob}[]{\vec{\mathbb{P}}}
\mathdef{\prob}[ O{q} ]{\mathbb{P}\{{#1}\}}
\mathdef{\cond}[ O{s} ]{\mathbb{P}\{\varepsilon_s|\vec{\gamma^r}\}}

\mathdef{\rep}[ O{i} O{r}]{\psi_{#1}^{#2}}


% TODO retester mathbb avec ACM. 
% Glossary content
% ---------------------------------------------------------------------------- %
% -  Thesis Glossary                                                         - %
% ---------------------------------------------------------------------------- %

%
% Acronyms
% 
% -- Examples:
% \newacronym{rssi}{RSSI}{Received Signal Strength Indication}
% \newacronym[plural=AP, longplural=Access Points]{ap}{AP}{Access Point}
% \newacronym[longplural=besoins non-exprimés]{bne}{BNE}{besoin non-exprimé}
\newignoredglossary{hidden}


% General
\newacronym[longplural={information technologies}]{it}{IT}{information technology}
\newacronym[longplural={operational Technologies}]{ot}{OT}{operational technology}
\newacronym{sdn}{SDN}{software-defined networking}
\newacronym{hr}{HR}{human resources}

% Domains
\newacronym{av}{AV}{autonomous vehicles}
\newacronym{gnb}{gNB}{next generation node base station}


% Mobile networls

% Security
\newacronym{ids}{IDS}{intrusion detection system}
\newacronym{nids}{NIDS}{network-based intrusion detection system}
\newacronym{hids}{HIDS}{host-based intrusion detection system}
\newacronym{ddos}{DDoS}{distributed denial of service}
\newacronym{dos}{DoS}{denial of service}
\newacronym{cia}{CIA}{confidenciality, integrity, availability}
\newacronym{waf}{WAF}{web application firewall}
\newacronym{ips}{IPS}{intrusion prevention system}
\newacronym{aaa}{AAA}{authentication, authorization, and accounting}
\newacronym{siem}{SIEM}{security information management system}
\newacronym{tee}{TEE}{trusted execution environment}
\newacronym{mtd}{MTD}{moving-target defense}
\newacronym{cids}{CIDS}{collaborative intrusion detection system}
\newacronym{mpc}{MPC}{multi-party computation}
\newacronym{soc}{SOC}{security operation center}
\newacronym{ad}{AD}{anomaly detection}
\newacronym{md}{MD}{misuse detection}
\newacronym{xss}{XSS}{cross-site scripting}
\newacronym{mitm}{MitM}{man-in-the-middle}

% Architectures
\newacronym{mape-k}{MAPE-K}{monitor-Analyze-plan-execute plus knowledge}
\newacronym{spof}{SPoF}{single point-of-failure}
\newacronym{smgw}{SMGW}{secure mediation gateway}
\newacronym{p2p}{P2P}{peer-to-peer}
\newacronym{mec}{MEC}{mobile edge-computing}
\newacronym{stin}{STIN}{satellite-terrestrial integrated network}
\newacronym{vm}{VM}{virtual machine}

% Networking
\newacronym{vpn}{VPN}{virtual private network}
\newacronym{mud}{MUD}{manufacturer usage description}
\newacronym{iat}{IAT}{inter-arrival time}

% I/IoT
\newacronym{ics}{ICS}{industrial control system}
\newacronym[plural=IoTs]{iot}{IoT}{Internet of Things}
\newacronym{iiot}{IIoT}{industrial Internet of Things}
\newacronym{cps}{CPS}{cyber-physical system}
\newacronym{mas}{MAS}{multi-agent systems}
\newacronym{momas}{MOMAS}{multi-objective multi-agent system}
\newacronym{bms}{BMS}{building management system}
\newacronym{plc}{PLC}{programmable logic controller}
\newacronym{siot}{SIoT}{social IoT}
\newacronym{ms}{\(\mu\)S}{microservice}
\newacronym{manet}{MANET}{mobile ad-hoc network}

% CTI
\newacronym[longplural={techniques, tactics, and procedures}]{ttp}{TTP}{techique, tactic, and procedure}
\newacronym{tti}{TTI}{technical threat intelligence}
\newacronym{cti}{CTI}{cyber threat intelligence}
\newacronym{ti}{TI}{threat intelligence}
\newacronym{apt}{APT}{advanced persistent threat}
\newacronym[plural=IoCs, longplural=Indicators of Compromise]{ioc}{IoC}{indicator of compromise}
\newacronym{stix}{STIX}{Structured Threat Information Expression}
\newacronym{taxii}{TAXII}{Trusted Automated eXchange of Indicator Information}


% Math and stats
\newacronym{iid}{IID}{independent and identically distributed}
\newacronym{niid}{non-IID}{non-independent and identically distributed}
\newacronym{pca}{PCA}{Principal Component Analysis}

% ML
\newacronym{ml}{ML}{machine learning}
\newacronym{ai}{AI}{artificial intelligence}
\newacronym{dm}{DM}{Decision-Making}
\newacronym{ddm}{DDM}{distributed decision-making}
\newacronym{svm}{SVM}{Support Vector Machine}
\newacronym{lsvm}{LSVM}{Linear Support Vector Machine}
\newacronym{cdw}{CDW}{centroid distance weighted}
\newacronym{rf}{RF}{random forest}
\newacronym{gmm}{GMM}{Gaussian mixture model}
\newacronym{sgd}{SGD}{stochastic gradient descent}
\newacronym{gd}{GD}{gradient descent}
\newacronym{lr}{LR}{logistic regression}
\newacronym{birch}{BIRCH}{balanced iterative reducing and clustering using hierarchies}
\newacronym{tl}{TL}{transfer learning}
\newacronym{mtl}{MTL}{multi-task learning}
\newacronym{elm}{ELM}{Extreme Learning Machine}
\newacronym{rl}{RL}{reinforcement learning}
\newacronym{dt}{DT}{decision tree}
\newacronym{knn}{KNN}{K-nearest neighbors}



% DL
\newacronym{dl}{DL}{deep learning}
\newacronym{drl}{deep RL}{deep reinforcement learning}
\newacronym{sda}{SDA}{stacked denoising autoencoders}
\newacronym{nn}{NN}{neural network}
\newacronym{dnn}{DNN}{deep neural network}
\newacronym{cnn}{CNN}{convolutional neural network}
\newacronym{rnn}{RNN}{recurrent neural network}
\newacronym{mlp}{MLP}{multilayer perceptron}
\newacronym{gru}{GRU}{gated recurrent unit}
\newacronym{gan}{GAN}{generative adversarial networks}
\newacronym{sfl}{SFL}{segmented federated learning}
\newacronym{dqn}{DQN}{deep Q-network}
\newacronym{bnn}{BNN}{Binarized Neural Network}


% FL
\newacronym{fl}{FL}{federated learning}
\newacronym{cdfl}{CD-FL}{cross-device federated learning}
\newacronym{csfl}{CS-FL}{cross-silo federated learning}
\newacronym{fml}{FML}{federated mimic learning}
\newacronym{hfl}{HFL}{horizontal federated learning}
\newacronym{vfl}{VFL}{vertical federated learning}
\newacronym{ftl}{FTL}{federated transfer learning}

% Evaluation
\newacronym{tpr}{TPR}{true positives rate}
\newacronym{fpr}{FPR}{false positives rate}
\newacronym{tnr}{TNR}{true negative rate}
\newacronym{fnr}{FNR}{false negative rate}
\newacronym{ppv}{PPV}{positive predictive value}
\newacronym{roc}{ROC}{receiver operating characteristic}
\newacronym{mcc}{MCC}{Matthews correlation coefficient}
\newacronym{mse}{MSE}{mean squared error}

% Distributed Leadgers
\newacronym{auc}{AUC}{area under the curve}
\newacronym[longplural=distributed ledger technologies]{dlt}{DLT}{distributed ledger technology}
\newacronym{dag}{DAG}{directed acyclic graph}

% Technical solutions
\newacronym{swat}{SWaT}{Secure Water Treatment}
\newacronym{iads}{IADS}{Intrusion and Anomaly Detection System}
\newacronym{vsl}{VSL}{Virtual State Layer}
\newacronym{rsu}{RSU}{roadside unit}
\newacronym{v2x}{V2X}{vehicle-to-everything}

% Conferences and journals
\newacronym[type=hidden]{icccn}{ICCCN}{IEEE International Conference on Computer Communications and Networks}
\newacronym[type=hidden]{icdcs}{ICDCS}{IEEE International Conference on Distributed Computing Systems}
\newacronym[type=hidden]{cnsm}{CNSM}{International Conference on Network and Service Management}
\newacronym[type=hidden]{icecce}{ICECCE}{International Conference on Electrical, Communication, and Computer Engineering}
\newacronym[type=hidden]{bdsa}{BigDataSE}{IEEE International Conference on Big Data Science and Engineering}
\newacronym[type=hidden]{iccc}{ICCC}{IEEE/CIC International Conference on Communications in China}

% University and other research institutes
\newacronym{accs}{ACCS}{Autralian Center for Cyber Security}
\newacronym{cic}{CIC}{Canadian Institute of Cybersecurity}
\newacronym{cse}{CSE}{Communication Security Establishment}

% Research and methodology
\newacronym{slr}{SLR}{Systematic Literature Review}
\newacronym{rq}{RQ}{research question}

% Own
\newacronym{fids}{FIDS}{Federated Intrusion Detection System}


%
% Glossary
%
% -- Note: adding '%' at the end allows to break line without semantic
%
% -- Examples:
% \newglossaryentry{wifi}
% {
%     name=WI-FI,
%     plural=WI-FI,
%     description={
%         Technologie sans-fil qui permet à des appareils de communiquer entre
%         eux, notamment sur Internet%
%     }
% }




%%%%%%%%%%%%%%%%%%%%%%%%%%%%%%%%%%%%%%%%%%%%%%%%%%%%%%%%%%%%%%%%%%%%%%%%%%%%%%%%
%               DOCUMENT
%%%%%%%%%%%%%%%%%%%%%%%%%%%%%%%%%%%%%%%%%%%%%%%%%%%%%%%%%%%%%%%%%%%%%%%%%%%%%%%%

%%
%% end of the preamble, start of the body of the document source.
\begin{document}

%% The "title" command has an optional parameter,
%% allowing the author to define a "short title" to be used in page headers.
\title[% short version
    Filtering Contributions in Federated Learning for Intrusion Detection
]{%
    Filtering Contributions in Federated Learning for Intrusion Detection: a Cross-evaluation approach for Reputation-aware Model Weighting
}

% TODO: choose final title:
% - Client Trustworthiness Assessment in Highly Heterogeneous Federated Learning Settings for Intrusion Detection
% - 


%% The "author" command and its associated commands are used to define
%% the authors and their affiliations.
%% Of note is the shared affiliation of the first two authors, and the
%% "authornote" and "authornotemark" commands
%% used to denote shared contribution to the research.

\newcommand{\redactedauthor}{%
    \author{Redacted for double-blind}
    \email{redacted@redacted.com}
    \affiliation{%
        \institution{REDACTED}
        \institution{REDACTED}
        \city{Redacted}
        \country{Redacted}
    }
}

\redactedauthor
\redactedauthor
\redactedauthor
\redactedauthor
\redactedauthor
\redactedauthor
\redactedauthor
\redactedauthor

%\author{Léo Lavaur}
%\authornote{Both first authors contributed equally to this research.}
%\email{leo.lavaur@imt-atlantique.fr}
%% \orcid{1234-5678-9012}
%\affiliation{%
%  \institution{IMT Atlantique}
%  \institution{IRISA / SOTERN}
%  \institution{Chair CyberCNI}
%  \city{Rennes}
%  \country{France}
%}
%
%\author{Pierre-Marie Lechevalier}
%\authornotemark[1]
%\email{pierre-marie.lechevalier@imt-atlantique.fr}
%\affiliation{%
%  \institution{IMT Atlantique}
%  \institution{IRISA / ADOPNET}
%  \city{Rennes}
%  \country{France}
%}

%\author{Fabien Autrel}
%\email{fabien.autrel@imt-atlantique.fr}
%\affiliation{%
%  \institution{IMT Atlantique}
%  \institution{IRISA / SOTERN}
%  \city{Rennes}
%  \country{France}
%}
%
%\author{Yann Busnel}
%\email{yann.busnel@imt-nord-europe.fr}
%\affiliation{%
%  \institution{IMT Atlantique}
%  \institution{IRISA / SOTERN}
%  \city{Rennes}
%  \country{France}
%}
%
%\author{Hélène Le~Bouder}
%\email{helene.le-bouder@imt-atlantique.fr}
%\affiliation{%
%  \institution{IMT Atlantique}
%  \institution{IRISA / OCIF}
%  \city{Rennes}
%  \country{France}
%}
%
%\author{Romaric Ludinard}
%\email{romaric.ludinard@imt-atlantique.fr}
%\affiliation{%
%  \institution{IMT Atlantique}
%  \institution{IRISA / SOTERN}
%  \city{Rennes}
%  \country{France}
%}
%
%\author{Marc-Oliver Pahl}
%\email{marc-oliver.pahl@imt-atlantique.fr}
%\affiliation{%
%  \institution{IMT Atlantique}
%  \institution{IRISA / SOTERN}
%  \institution{Chair CyberCNI}
%  \city{Rennes}
%  \country{France}
%}
%
%\author{Géraldine Texier}
%\email{geraldine.texier@imt-atlantique.fr}
%\affiliation{%
%  \institution{IMT Atlantique}
%  \institution{IRISA / ADOPNET}
%  \city{Rennes}
%  \country{France}
%}


%%
%% By default, the full list of authors will be used in the page
%% headers. Often, this list is too long, and will overlap
%% other information printed in the page headers. This command allows
%% the author to define a more concise list
%% of authors' names for this purpose.
%\renewcommand{\shortauthors}{Lavaur and Lechevalier, et al.}
\renewcommand{\shortauthors}{Redacted and Redacted, et al.}

\begin{abstract}
    % (blue) Basic introduction to the field (one or two sentences): 
    %\Glspl{cids} improve resilience to distributed attack campaigns by allowing \acrshortpl{ids} to share locally acquired knowledge with peers.
    %While it consequently improves detection, data sharing is challenging, especially in \gls{ml} workflows.
    % (pink) More detailed background (Two or three s)
    
    \Gls{fl} is a distributed learning paradigm that enables \glspl{cids} to benefit from the experience of several participants without disclosing their local data. 
    % \Gls{fl} is a distributed learning paradigm that enables \glspl{cids} without the need to share local data, while still benefiting from the experience of all participants.
    %By sharing the learning task among several actors, it allows training a global intrusion detection model that benefits from the data of all participants.
    Negligent or malicious clients might, however, negatively contribute to the global model and degrade its performance.
    % (yellow) General problem (1 s)
    While existing approaches typically identify negative contributions through model comparison, participants in \gls{cids} are inherently heterogeneous. 
    Thus, it is difficult to differentiate a malicious update from a legitimate one coming from a different dataset.
    % Existing approaches to detect adversaries tend to falter in heterogeneous settings, while \gls{cids} federations are inherently heterogeneous.
    
    In this paper, we present a novel \gls{fl} architecture for intrusion detection, able to deal with both, heterogeneous and malicious contributions, without the need for a single source of truth.
    We leverage client-side evaluation for clustering participants based on their perceived similarity, and then feed these evaluations to a reputation system that weights participants' contributions based on their trustworthiness.

    We evaluate our approach against four intrusion detection datasets, in both benign and malicious scenarios. 
    % (cyan) Summarizing the main results (one s)
    We show that our clustering successfully groups participants originating from the same dataset together, while excluding the noisiest attackers. 
    The reputation system then strongly limits the impact of the stealthier ones within each cluster, as long as they remain a minority.
    
    % (green) summarizing how the result/approach compares with past work (2 or 3 s)
    %We confront our work with a state-of-the-art poisoning mitigation approach. 
    %We show comparable performance when using a single dataset, and vastly outperform them when data originate from multiple datasets, a case we deem more realistic for \gls{cids}.  

    The comparison of our work with a state-of-the-art mitigation strategy highlights its versatility.
    In particular, we outperform them on \acrshort{iid} and practical \acrshort{niid} use cases, while remaining comparable in the pathological \acrshort{niid} ones, that are less relevant for \gls{cids}.
    
    % (purple) Put the results in a more general context (1 or 2)
    % (red) optional perspective (2 or 3)
    % Nice but not relevant for abstract
    % Our work is open source, and all code can be found at: \codeurl. 
\end{abstract}

% (blue) Basic introduction to the field (one or two sentences): 
    %\Glspl{cids} improve resilience to distributed attack campaigns by allowing \acrshortpl{ids} to share locally acquired knowledge with peers.
    %While it consequently improves detection, data sharing is challenging, especially in \gls{ml} workflows.
    % (pink) More detailed background (Two or three s)
    %\Gls{fl} distributes the learning task among several actors, and allows training a global intrusion detection model that benefits from the data of all participants.
    %Negligent or malicious clients might, however, negatively contribute to the global model and degrade its performance.
    % (yellow) General problem (1 s)
    %While existing approaches typically identify negative contributions through model comparison, participants in \gls{cids} are inherently heterogeneous. 
    %Thus, it is difficult to differentiate a malicious update from a legitimate one coming from a different dataset.
    
    %In this paper, we present a modified \gls{fl} framework for intrusion detection, able to deal with both, heterogeneous and malicious contributions, without the need for a single source of truth.
    %We leverage client-side evaluation for clustering participants based on their perceived similarity, and then feed these evaluations to a reputation system that weights participants' contributions based on their trustworthiness.

    %We evaluate our approach against four intrusion detection datasets, in both benign and malicious scenarios. 
    % (cyan) Summarizing the main results (one s)
    %We show that the clustering method successfully group participants originating from the same dataset together, while excluding the noisiest attackers. 
    %The reputation system then strongly limits the impact of sneakier attackers within each cluster, as long as they remain a minority.
    
    % (green) summarizing how the result/approach compares with past work (2 or 3 s)
    %We confront our work to a state-of-the-art poisoning mitigation approach.  We show comparable performance on data coming from a single dataset, and vastly outperform them when data originate from multiple datasets, a case we deem more relevant for \gls{cids}.  
    
    % (purple) Put the results in a more general context (1 or 2)
    % (red) optional perspective (2 or 3)
    %Our work is open source, and all code can be found at: \codeurl. 

\glsresetall % reset all previous glossary calls

%%
%% The code below is generated by the tool at http://dl.acm.org/ccs.cfm.
%%
\begin{CCSXML}
<ccs2012>
   <concept>
       <concept_id>10002978.10002997.10002999</concept_id>
       <concept_desc>Security and privacy~Intrusion detection systems</concept_desc>
       <concept_significance>500</concept_significance>
       </concept>
   <concept>
       <concept_id>10002978.10003006.10003013</concept_id>
       <concept_desc>Security and privacy~Distributed systems security</concept_desc>
       <concept_significance>500</concept_significance>
       </concept>
   <concept>
       <concept_id>10010147.10010257.10010258.10010260.10010229</concept_id>
       <concept_desc>Computing methodologies~Anomaly detection</concept_desc>
       <concept_significance>300</concept_significance>
       </concept>
 </ccs2012>
\end{CCSXML}

\ccsdesc[500]{Security and privacy~Intrusion detection systems}
\ccsdesc[500]{Security and privacy~Distributed systems security}
\ccsdesc[300]{Computing methodologies~Anomaly detection}


%
%
%% Keywords. The author(s) should pick words that accurately describe
%% the work being presented. Separate the keywords with commas.
\keywords{federated learning, intrusion detection, reputation systems, trust,
heterogeneity, clustering, cross-evaluation}

 \received{March, 29 2023}
\received[revised]{--}
\received[accepted]{--}

%%
%% This command processes the author and affiliation and title
%% information and builds the first part of the formatted document.
\maketitle

% Sections
% ------------------------------------------------------------------------------



\section{Introduction}\label{sec:intro}

%With the ever-increasing number of cyberattacks, \gls{ids} play a critical role in the protection of information systems.
%\Glspl{nids} are especially relevant for organizations, as they can detect malicious activities in large-scale information networks, such as data exfiltration or \gls{dos} attacks. 
%
%\Gls{ml} is significantly represented in the \gls{nids} literature, enabling the characterization of network traffic and detection of abnormal behaviors.
%However, these \gls{ml} algorithms require large quantities of data to be trained.
%While data-sharing between organizations could address this requirement, 
%As a result, stakeholders generally show reluctance to adopt \glspl{cids}.

For several years, the adoption of collaborative \gls{ml} has been limited by the risks induced in data-sharing.
\Glspl{nids} are especially concerned, as sharing network data can expose information about the inner workings of information systems.
The recent advances~\cite{kairouz_advances_2021} in \gls{fl} promise to solve such issues, allowing participants to collaboratively train a global model without sharing their local data~\cite{mcmahan_communication-efficient_2017}.
% - Contexte de la contribution et cas d'application du FL: conglomérat d'organisations (on cite acteurs industriels ?) ayant des exigeances de sécurité suffisament proche pour souhaiter partager des informations de détection d'intrusion.
%\Gls{fl} is a privacy-preserving distributed learning paradigm that allows 
Specifically, organizations in \glspl{cids} can leverage \gls{hfl} (\ie same features, but different samples), to share their observations with other participants while keeping their locally collected network data private~\cite{lavaur_evolution_2022}. %, thus virtually extending the size of their training set.

% - PB1: Tout systeme collaboratif s'expose à la participation d'acteurs malveillants ou simplement de bonne foi avec des données de mauvaise qualité -> créer de la confiance
Collaborative systems are especially sensitive to input quality.
For instance, malicious participants in a \gls{cids} could poison their contributions to impact the convergence of the global model, or introduce backdoors that could be exploited afterward.
Even honest participants can negatively contribute to the aggregation by training their model on data of poor quality or by being unaware of attacks present in their network.  
% - PB2: Dans un contexte fortemmet hétérogène il est difficile de distinguer une attaque d'une contribution différente du fait de ses données d'entrées.
Moreover, different organizations might exhibit substantial differences in their information systems, such as hosted services or used protocols.
This can lead to significant variations in model updates, as each organization trains its model on its local network traffic.
In this heterogeneous context, it is very difficult to distinguish a negative contribution from a legitimate one originating from a different infrastructure.

% SOA:
% -
% Approaches that aim to mitigate model poisoning typically leverage a single source of truth or model comparisons .
% They either compare  on a single source of truth \cite{blanchard_machine_2017,cao_fltrust_2022}, such as a server-maintained model or a representative training set, to compare participants' updates with.
% However, building a single source of truth is infeasible in \gls{niid} settings, due to the differences between participants.
% % - en mode "pair-wise"
% In contrast, other works compare participants' model updates with each other.
% It can be used to either remove outliers in \gls{iid} settings~\cite{yin_byzantine-robust_2018}, or detect participants that collude to poison the global model in \gls{niid} settings~\cite{fung_limitations_2020, awan_contra_2021}. 
% These \gls{niid} technics, however, fail to detect a single attacker, we also show that in case where legitimate participant look alike they can be detected as colluding participants. 

% - anti-poisoning iid
Approaches that mitigate model poisoning in homogeneous distributions typically compare or evaluate a model using a single source of truth~\cite{blanchard_machine_2017,cao_fltrust_2022}.
% KO en non-iid
Building such a single source of truth, however, is infeasible in heterogeneous contexts due to the differences between participants. 
Assuming that all contributions are different, some approaches detect colluding attackers based on their similarity~\cite{fung_limitations_2020, awan_contra_2021}. 
%By construction, 
Nevertheless, these approaches fail to detect an isolated, yet potent, attacker.
%We also show that in case where legitimate participant look alike, they can be unfairly detected as colluding participants. 


% % % - avec single source of truth
% Some rely on a single source of truth \cite{blanchard_machine_2017,cao_fltrust_2022}, such as a server-maintained model or a representative training set, to compare participants' updates with.
% However, building a single source of truth is infeasible in \gls{niid} settings, due to the differences between participants. 
% % - en mode "pair-wise"
% In contrast, other works compare participants' model updates with each other.
% It can be used to either remove outliers in \gls{iid} settings~\cite{yin_byzantine-robust_2018}, or detect participants that collude to poison the global model in \gls{niid} settings~\cite{fung_limitations_2020, awan_contra_2021}. 
% These \gls{niid} technics, however, fail to detect a single attacker, we also show that in case where legitimate participant look alike they can be detected as colluding participants. 
% However, similarity-based filtering would exclude model updates that are different, even though they are relevant to the global model.
% For instance, two models trained on different protocols can differ statistically, could maintain high accuracy on a common test set.


% % notre approache fait (présentation TRES rapide)
% % - les métriques de comparison de modèles entre eux ne sopnt pas pertyinentes (single source, model diff mais performant) -> cross-evaluation
% % - besoinde réduire l'hétérogénéité -> clustering (cite qques paiers qui en font déjà) mais en utilisant les métriques fournies par la xeval
% % - cluster ne prend pas enc compte l'historique, et risque de manquer des déviation fines -> réputation en prenant les eval comme feedback
% % - objectif: construire de la confiance par éval sucessive
% For these reasons, we present \thecontrib, a defense strategy for \gls{csfl} that relies on client-side model evaluation.
% Instead of comparing participant models on the server, we propose that participants subjectively evaluate the others' contributions. %' quality. 
% To that end, we add a new step to the typical \gls{fl} workflow, between training and aggregation, where each participant uses its local dataset to evaluate the other participants' models. 
% %Based on the result of this cross-evaluation, our approach regroups similar-looking participants together using hierarchical clustering, before weighting their model aggregation using a reputation system.
% %These evaluations are adjusted based on their historical similarity with other cluster member evaluations. 
% Our approach then leverages this \emph{cross-evaluation} to assess participants' similarity and group them using hierarchical clustering, before weighting the model aggregation using a reputation system.

In this paper, we present \thecontrib, a defense strategy for \gls{csfl} aiming at detecting attackers (colluding or not) regardless of the data homogeneity. \thecontrib relies on three main ingredients:
\begin{enumerate*}[label=\em {\roman*})]
    % \item contrarily to classical FL approaches where participants models are compared on a centralized server, we propose that participants subjectively evaluate the others' contributions. %' quality. 
% To that end, we add a new step to the typical \gls{fl} workflow, between training and aggregation, where each participant uses its local dataset to evaluate the other participants' models,
    \item a modified \gls{fl} workflow, where each participant uses its local dataset to evaluate the other participants' models, between training and aggregation steps, in contrast with classical FL approaches where participants models are compared on a centralized server,
    % \item a modified \gls{fl} workflow, where each participant evaluates the other participants' models thanks to its local dataset,
    \item participants are gathered into clusters thanks to their similarity,
    \item a reputation system is leveraged to weight participants contributions.
    %on a global model.
\end{enumerate*}

% Our evaluation shows that \thecontrib can effectively identify lone and colluding attackers in highly heterogeneous settings, while ensuring convergence of the benign participants.
We measure the performance of \thecontrib using four network flow datasets with standardized features, representing various use cases.
%We show that our clustering method can reproduce the initial distribution of participants. 
We also compare our approach to existing ones~\cite{mcmahan_communication-efficient_2017,fung_limitations_2020}, and find that \thecontrib detects attackers under most attack scenarios, including lone and colluding attackers, targeted and untargeted attacks, in different types of poisoning strategies.

% Summary
To summarize, our contributions are threefold:
%TODO LN moi je metterai des formes passives
\begin{enumerate}[label=({\arabic*})]
    \item we present \thecontrib, an architectural scheme to protect \gls{fl} strategies using clustering and reputation-aware aggregation; with an extensive evaluation under different attacks and distributions, and against relevant baselines,
    \item we show that evaluation metrics (such as accuracy, f1-score, or loss) can effectively be used to assess similarity between \gls{fl} participants, and as input to clustering algorithms,
    \item we provide a \gls{fl} reputation system, leveraging local evaluation to swiftly adapt aggregation weights and protect the system's convergence.
    %\item We compare our approach with \texttt{FoolsGold}~\cite{fung_limitations_2020}, a relevant baseline that emphasizes on detecting colluding attackers in heterogeneous contexts.
    %\item We provide extensive evaluation \thecontrib under different types of attacks and client distributions.
\end{enumerate}

%%%% §7 Leo
The rest of this paper is organized as follows.
\Cref{sec:problem} defines the addressed problem and introduces our threat model.
We detail in \Cref{sec:related} our positioning and \thecontrib's benefits over related works.
\Cref{sec:archi} presents \thecontrib's design and core components, before an extensive evaluation in \Cref{sec:eval}.
We discuss our findings in \Cref{sec:discussion} before concluding and laying out future works.




% NIDS background
%Intrusion detection refers to methods and systems that can identify potential threats in an information system.
%Sharing actionable intelligence between organizations can improve the performance of \glspl{ids}. 
%However, collaboration also brings challenges, \eg trust between parties, privacy and confidentiality and, finally, protection against poisoning~\cite{wagner_novel_2018}.
%
%Typical collaborative \gls{ml} requires all data to be centrally collected, curated, and processed.
%Sending training data to a remote server implies sending it over a network.
%Depending on the type of training data, it can represent an important bandwidth consumption, especially for \gls{nids} workflows.
% Finally, it can also mean that clients are unable to exploit their data locally, and solely rely on a remote server to provide them with t













%%%%%%%%%%%%%%%
%%%%%%%%%%%%%%%
%%%%%%%%%%%%%%%

% By only sharing the model weights instead of the underlying data or IoC federated learning limit the risk of unintentional information disclosure. 
% Traditionally, and for the sake of simplicity, the aggregation of the global model is done using the average weight from local models.
% Each local model from the participants thus have an equal impact on the global model which might not be ideal.%% TODO 2 : insérer citation, le nom de "l'algo" de base c'est FedAvg ?
% In our case, different participants might have infrastructures too diverse to make knowledge transfer between participants possible. 
% Also, compromised participants might unknowingly send inaccurate classification that end up poisoning the aggregated model.  
% Strategies that take into account the quality of the local model during the aggregation process have been shown to increase the accuracy of the global model and to converge faster.\cite{wang_reputation-enabled_2021,wang_novel_2020}

% Theses strategies usually rely on a comparison between the evaluated local model and one or multiple other models. 

%The comparison can take different forms : 
% \begin{enumerate}
%     \item It can be made against variants from the global model, e.g. the global model from the last round. Alternatively, against a temporary version of the new global model curated from the evaluated local model. \cite{wang_reputation-enabled_2021} \cite{xia_tofi_2021}
%     \item It can be a pairwise comparison with the others local models \cite{wang_flare_2022} 
%     \item Deviation from historical local models
%     \item Assessing the accuracy of the local models by testing it against a data-set located on the central actor that is in charge of the aggregation process.\cite{cao_fltrust_2022} \cite{xia_tofi_2021}. 
% \end{enumerate}
% 
% This last strategy however make the assumption that the central actor detain a part of the data set that is representative of the locals data set and are typically more aimed at IID-dataset.
% 
% In order to also address non IID-dataset, having a representative data set located on central actor is too strong of an assumption.
% 
% We try to tackle this problem by letting the participants evaluate the local models accuracy of their peer themselves. 
% 
% Based on existing work in trust and reputation we build a trust overlay to better judge the feedback that participants sent on the local models of their peers. 
% 
% The reliability of this approach is dependant on the collaboration between participants and thus on the chosen threat model that will be detailed in section TODO. 
% % TODO : lien vers la section threat model
% Aggregation strategies usually discard local models that are too different from the global model.
% Based on the hypothesis that different models might be more accurate for diverse environments we choose to cluster local models that match together instead of pruning all the deviations from the global model. 
% This effectively create several global models in which a participant local model can be aggregated. 
% 
% % Several strategies have also been suggested to measure the difference between two models : 
% % \begin{enumerate}
% %     \item cosine similarity 
% %     \item
% % \end{enumerate}
% 
% % Attention je n'ai pas mentionné l'élément différentiant qui est le fait de tester l'accuracy du modèle ches les participants plutôt que centralement ou de se baser sur un score similitude. 
% % C'est bénéfique car cela permet de ne pas avoir de données sur l'acteur centralisé et de traiter des cas non-IID. Nous sommes typiquement dans ce type de scénario car les organisations peuvent avoir des infrastructurs très différentes. 
% 
% 
% %\subsection{Use case of collaborative intrusion detection} 
% 
% The goal is to be able to detect an anomaly and an intrusion with several systems collaborating.
% We propose to use federated learning.
% 
% An entity has a global view of its network (with various devices), it has a knowledge of the traffic running on the network (pcap file for ex). 
% The goal is to learn the legacy comportment by learning it on the traffic, then it will be able to detect an anomaly (by reducing the dimension of the model). 
% It will compute the reduction of the dimension by exchanging information with other entities (organizations participating in the collaboration to detect anomalies).
% 
% The model is synchronous with rounds. 
% The system will ask the model to learn and to share its state with others involved in the collaboration.
% 
% At each round, the server shares the current model (the weight matrix). 


%%%
% Léo : détient des métriques pertinentes à observer pour savoir quelle serait l'overhead d'un tel système et les critères % de réussite.
% 
% Hypothèse :
% 	- 
% 	
% ## Evaluation système malveillant 
% - Consomation ressource 
% - Détection d'un acteur malveillant  
%  
%  Regarder plus précisement les métriques d'intérêt du point de vue du système de réputation : 
% - Acteurs malveillant. 
% - Consomation ressources.
% 
% 
% 
% Distiction de ce travail : 
% - vis à vis du feaderated learning
% - ou vis à vis de 
% 
% 
% Voir s'il y a particularité vis à vis des autres systèmes de réputation utilisés dans le federated learning ? 
% - Est-ce que l'aspect clusterisation. 
% - 
% 
% Clusterisation à partir de l'évaluation local du modèle. 
% Déviance du modèle vis à vis d'un historique. 
% Ecart du modèle : 
% 	- distance entre les deux matrices de poids. 
% 
% On peut vérifier la distance entre deux modèles parfaits. 
% 
% 
% Parcourir l'histoire du usecase : 
% 
% 
% Histoire pour le usecase :
% ### Point de vue du client
% Je (un appareil d'une organisation) reçoit des logs du traffic passant sur l'ensemble du réseau (qui peut être de taille variable) j'en apprend le comportement normal via réduction de dimmension : on trouve un modèle qui reconstruit le trafic et on observe l'écart entre le trafic observé et le traffic réel. Si il y a un écart c'est que c'est une anomalie. C'est e modèle de réduction qui est partagée sous forme d'une matrice de poids à d'autres organisations.

% ##C hoix A
% Synchronization par round synchrone dictés par le serveur de federated learning qui m'envoie le modèle courant (aléatoire à l'initialaisation, résultat du dernier round sinon) et des tâches de travail. Une fois que ma tâche de travail est terminée je renvoie mon modèle au serveur. 
% Le serveur me renvoie en retour des modèles à évaluer sur mon jeu de données. Je les laisses tourner sans m'en servir pour % faire de la classification et renvoi un score d'accuracy au serveur.
% 
% - clusterisation 
% - répartition des poids



% ## Choix B
% Possible de faire des systèmes moins sychrone avec du choix de l'information. 
% Les modèles sont pas très gros. 
% 
% 
% Une entitée 
% 
% Serveur centrale 
 
\input{sections/20_background.tex}
\input{sections/30_problem.tex}
\input{sections/40_soa}
\input{sections/50_archi}
\input{sections/60_eval}
\input{sections/80_conclusion}
\input{sections/90_ack}

%%%%%%%%%%%%%%%%%%%%%%%%%%%%%%%%%%%%%%%%%%%%%%%%%%%%%%%%%%%%%%%%%%%%%%%%%%%%%%%%
%               DOCUMENT END
%%%%%%%%%%%%%%%%%%%%%%%%%%%%%%%%%%%%%%%%%%%%%%%%%%%%%%%%%%%%%%%%%%%%%%%%%%%%%%%%

% \defaultprintbib % print the bibliography

\printbibliography

\end{document}
\endinput
%%
%% End of file `main.tex'.
