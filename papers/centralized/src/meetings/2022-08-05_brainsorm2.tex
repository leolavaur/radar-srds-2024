Système de réputation pour federated learning. 


Caractéristiques : 

### local trust / efficiency 
- Pas de votes mais des résultats d'efficacité continus pas discrets (aire sous la courbe roc à maximiser ?)	
	- Potentiellement des faux négatifs qui se comportent un peu comme des actions positives 
	- Types d'actions : 
		- True positive : récompense
		- False positive : pénalisation ? 
		- Faux négatifs : neutre, le modèle n'est pas fort mais n'est pas nocif non plus : peut susceptible d'arriver le modèle ne sera pas forcément remonté si il ne fait rien 
		- True négative comportement nominal, neutre

- Clustering : on mesure l'affinité que les clients ont entre eux ou celle qu'ils ont avec le modèle d'un "cluster" ? 
	
		
### Aggrégation 
- Aggrégation des résultats, notion de poids : 
	- Les clients en dehors du cluster ont un poids de 0 (ou très faible ?)
	- A l'intérieur du cluster toutes les contributions ne se valent pas forcément, on peut prendre en compte "l'apport en efficacité" pour augmenter le poids (ex: aire sous la courbe roc / aire de l'ensemble de des clients du cluster ?)  
		- Risque de converger vers un modèle unique ?
  
#### normalization 
- Dépend de la manière dont est constitué le modèle global.


### Trust propagation 
	non traité dans la contrib 1
		
### Trust decay : 
	- ré-évaluation périodique de l'appartenance des clients à leur cluster, au moins pour ceux qui n'auraient pas de cluster. 

### Nouvaux entrants :
	- Hors du scope de la contribution 1, les participants sont connus et identifiés.
	
### Quand est-ce que c'est le bon moment pour fusionner/séparer des cluster ? 
	- possible de ne pas traiter cela 
	- comparaison ponctuelle des éléments d'un cluster avec d'autres clusters
	  
### Collaboration incentive 
: federated learning, les participants reçoivent un modèle consolidé en retour de leur contribution.
	- Prevent free loading
		-> non adressé dans la collab 1 du fait du modèle de menace. 

### Détection de vote malveillants 
écarté du fait de l'architecture
	
## Hypothèses de travail contribution 1 : 
	- Les participants sont connus d'avance et identifiés
		- Problématiques d'entrée de nouveaux acteurs écartés en contrib 1.
	-Le modèle d'attaque est curieux et honnête, les acteurs peuvent être compromis mais sont de bonne foi
 
## Hypothèses de travail 


Est-ce que le modèle global est aggrégé de manière régulière/planifié avec des apports de l'ensemble des participants en même temps ou 
Dès qu'un client à des informations intéressantes il remonte son modèle ? 
Je ne connais pas vraiment le processus d'aggrégation.   
Peut-être sra-t-il intéressant à un stade de mesure l'overhead en communications 

