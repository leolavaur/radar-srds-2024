Points abbordés : 
- Existence d'un papier faisant de la cross evaluation des modèles locaux par les clients:  
    - Contribution proposé différencié par:
      - clustering des client pour contrer les problématiques d'hétérogénéité + les comportement négligeants
      - cas d'usage cross-silo = pas de contrainte du nombre de clients sur lesquels évaluer les modèles ni de nécessité de regrouper les modèles locaux en sous-modèles pour l'évaluation.
      - cas d'usage détection d'intrusion (priorisation non-supervisé -> pas de label-switching, etc...)
    -> géraldine: ce papier est plutôt une bonne nouvelle car ça appuie notre approche mais on peux facilement s'en démarquer
    

Définir (jeudi matin): 
- threat model 
- cas d'usage 
- contraintes (e.g. cross-silo ...).

Priorisation pour les prochaines semaines : 
- Prendre conaissance des papiers citants \cite{zhao_shielding_2020} pour compléter l'état de l'art et vérifier qu'une approche de cross evalution + clustering n'existe pas déjà
- Mettre en pause la rédaction (hormis les points définis jeudi matin) 
- Avancer sur l'implémentation. 
    - Motivations : 
        - Léo & PM ont un travail de rédaction à fournir pour C&ESAR
        - Via les recherches déjà faites + le complément l'état de l'art aura commencé à prendre forme. 
    