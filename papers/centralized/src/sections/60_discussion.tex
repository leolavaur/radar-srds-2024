\section{Discussion}\label{sec:discussion}

\thecontrib shows versatility in how it addresses the complex problem of identifying attackers in heterogeneous contexts.
However, the interpretation of the presented results is limited to our small-scale \gls{nids} use case.
In this section, we discuss the limitations and potential consequences of our architecture and propose research directions to close these gaps. 

\paragraph{Generalizability.} 

While the experiments are only conducted on intrusion detection datasets, \thecontrib's design could be used in different use cases regarding the following conditions: (1) parametric local models whose parameters can be aggregated using \gls{fl}, and (2) local testing sets and relevant metrics allowing participants to evaluate the others’ models.
Since the \gls{nids} use case induces a focus on malicious samples (\ie \emph{positive} values), we choose the F1-score as input for both the clustering and reputation, as it emphasizes on false positives and false negatives.
Other use cases might find using the loss of a model more relevant, especially for similarity measurements.
Further, \thecontrib's resilience against practical \gls{niid} settings makes it also relevant for safety objectives, where negligent participants can also exist.

\paragraph{Scalability and performance.}

The focus on small-scale collaboration (\ie a few dozens of participants) makes the overhead of the \emph{cross-evaluation} step (\Cref{sec:archi.xeval}) practical, and justifies the absence of performance-related metrics in this paper.
However, it makes \thecontrib in its current state unsuitable for highly distributed scenarios.
Each round, clients evaluate $|P|$ additional models, which scales linearly with the number of clients.
Two new communications are also introduced, one to send the models and one to collect the evaluations.
The size of the former also grows linearly with $|P|$, as the models of all participants must be distributed.
Likewise, we exclude execution-related performance evaluation such as training time, CPU overhead, or bandwidth consumption.
%We leave the feasibility of \thecontrib to future works, and emphasize on its ability to maintain high accuracy.
%This makes \thecontrib more suited for small-scale collaboration, such as our \gls{cids} use case.
It opens the way to interesting research directions on how to implement and scale \thecontrib while guarantying its properties.


% // Removed because it only restates the findings of the Results section
% \paragraph{Attack mitigation.} 

% Throughout this paper, we emphasize on \thecontrib's versatility, as it can cope with multiple attack scenarios simultaneously.
% However, a group of numerous attackers with evaluations close enough to benign participants might be grouped with them.
% If they represent the majority of the newly created cluster, they will be favored by the reputation system.
% Therefore, combining \thecontrib with additional mitigation strategies inside clusters, more geared toward the detection of numerous malicious participants, is a relevant research direction.
\paragraph{Evaluation poisoning.}
%While not covered in the experiments, 
Attackers could try to poison the evaluations that they provide on other participants to abuse the system.
%One possibility for attackers would be to change arbitrarily the evaluations that they provide on other members to abuse the system. 
However, the implementation presented in \Cref{sec:eval.methodo} implies that attackers poison both their training and testing sets.
Consequently, the evaluations they produce on other participants are directly affected.
We thus expect the system to cope with arbitrary poisoning similarly to data poisoning: either by placing the attackers in a different cluster because of their dissimilarity, or by penalizing their reputation. 

% \paragraph{Information disclosure.}
% Because \thecontrib shares models with the other participants to obtain feedbacks, it can be argued that it revels more information about the participants.
% This is limited to the participants' models, which are shared without identifiers.
% However, since clients also receive the global model of their cluster, they can try to estimate the models that belong to their cluster.
% This remains challenging, as the models are weighted using the reputation score of the participants, which are only available to the server.
% Comparing the privacy impact of \thecontrib with those of simpler approaches like \texttt{FedAvg} represents interesting research directions.
